\documentclass{jsarticle}
\usepackage[dvipdfmx]{graphicx}
\usepackage[dvipdfmx]{hyperref}
\usepackage[dvipdfmx]{color}
\usepackage{float}
\usepackage{tikz}
\usepackage{ascmac,amsmath,amsfonts,amsthm,amssymb}
\usepackage{url}
\usepackage[compat=1.1.0]{tikz-feynhand}
\usepackage{empheq}
\usepackage{bm}
% general
\newcommand{\thus}{\raisebox{.2ex}{.}\raisebox{1.2ex}{.}\raisebox{.2ex}{.}\qquad} 
\newcommand{\floor}[1]{\lfloor #1 \rfloor}
\newcommand{\norm}[1]{ \left|\left| #1 \right|\right| }
\newcommand{\vb}[1]{\mathbf{#1}}
\newcommand{\abs}[1]{{\left| #1 \right|}}
\newcommand{\expval}[1]{\langle #1 \rangle}
\newcommand{\grad}[1]{\mathrm{grad}\ #1}                    
\renewcommand{\div}[1]{\mathrm{div}\ #1}
\newcommand{\rot}[1]{\mathrm{rot}\ #1}
\newcommand{\pdv}[2]{\frac{\partial #1}{\partial #2}}       %偏微分
\newcommand{\dv}[2]{\frac{\mathrm{d} #1}{\mathrm{d} #2}}    %微分
\newcommand{\ldv}[2]{\frac{\mathrm{D} #1}{\mathrm{D} #2}}   %ラグランジュ微分
\newcommand{\fdv}[2]{\frac{\delta #1}{\delta #2}}           %汎関数微分
\newenvironment{itmbx}[1]{\begin{itembox}[l]{#1}}{\end{itembox}}
\newenvironment{warn}{\color{red}\begin{itmbx}{\textcolor{black}{注意}}\color{black}}{\end{itmbx}}
\newcommand{\skakko}[1]{{\left(#1\right)}}                  %()
\newcommand{\mkakko}[1]{{\left[#1\right]}}                  %[]
\newcommand{\lkakko}[1]{{\left\{#1\right\}}}                %{}
\newcommand{\convination}[2]{\left(\begin{array}{c}{#1}\\{#2}\end{array}\right)}
\newcommand{\jacobiid}[3]{\mkakko{#1,\mkakko{#2,#3}}+\mkakko{#2,\mkakko{#3,#1}}+\mkakko{#3,\mkakko{#1,#2}}}
\newcommand{\mspan}[1]{ \mathrm{Span}\lkakko{#1} }
\newcommand{\T}{{\mathrm{T}}}                               %時間順序
\newcommand{\tr}[1]{\mathrm{Tr}\lkakko{#1}}                 %Tr
\newenvironment{calc}{\color{gray}}{\color{blue}}
\newcommand{\mel}[3]{\langle #1 | #2 | #3 \rangle}
\newcommand{\ket}[1]{| #1 \rangle}
\newcommand{\bra}[1]{\langle #1 |}
\newcommand{\braket}[2]{\langle #1 | #2 \rangle}
\newcommand{\op}[1]{\hat{\mathcal{#1}}}                     %作用素
\newcommand{\asin}{\mathrm{Arcsin}}
\newcommand{\acos}{\mathrm{Arccos}}
\newcommand{\atan}{\mathrm{Arctan}}
\newcommand{\asinh}{\mathrm{Arcsinh}}
\newcommand{\acosh}{\mathrm{Arccosh}}
\newcommand{\atanh}{\mathrm{Arctanh}}
\renewcommand{\Re}{\mathfrak{Re}}
\renewcommand{\Im}{\mathfrak{Im}}
\newcommand{\convcirc}{{\ \raisebox{-.5ex}{,}\mspace{-7mu}\raisebox{.5ex}{$\circ$}\ }}
\theoremstyle{definition}
\newtheorem{theorem}{定理}
\newtheorem*{theorem*}{定理}
\newtheorem{definition}[theorem]{定義}
\newtheorem*{definition*}{定義}
\newtheorem{lemma}[theorem]{補題}
%%%%%%%%%%%%%%%%%%%%%%%%%%%%%%%%%%%%%%%%%%%%%%%%%%%%%%%%%%%%%%%%%%%%%%%%%%%%%%%%%%%%%%%%%%%%%%%%%%%%%%




\title{Turing反応}
\author{Shumpei, Mao}
\date{\today}

\begin{document}
\maketitle

\setcounter{tocdepth}{3}
\tableofcontents


\part{Intro}\label{Intro}
今回やりたいレシピは
\begin{itemize}
    \item Turing反応のシミュレーションをJuliaでできるようにする
    \item 高速化
    \item 初期分布、パラメータ、モデルと一定ステップ後の分布をセットにしたデータセットを作る
    \item 適切な手法で学習させる
    \item 人間の発生段階でのパターン形成や、現実には取り扱いにくい物質と既存の物質のセットに関して拡散係数や反応経路を推測する
\end{itemize}

\part{原理論}\label{Theory}
Turing反応式の一般式
\begin{align}\label{Turing_reaction_eq}
    \frac{du}{dt}=&D_u \nabla^2 u+f(u,v)\\
    \frac{dv}{dt}=&D_v \nabla^2 v+g(u,v)
\end{align}
であり、fやgを変化させることによってモデルを区別する。
\subsection{FitzHung-Nagumo}\label{FitzHung-Nagumo}
FitzHugn-南雲モデルでは
\begin{align*}
    f(u,v)=&u-u^3-v\\
    g(u,v)=&\gamma\skakko{u-\alpha v-\beta }
\end{align*}
であり、解析を行うにはまず、線形近似を行う。非線形項$f,g$に対して$f\skakko{\tilde{u},\tilde{v}}=0,g\skakko{\tilde{u},\tilde{v}}=0$
となる定数$\tilde{u},\tilde{v}$を考えると、その周りにおいて
\begin{align*}
    u=&\tilde{u}+ae^{i\skakko{kr-\omega t}}\\
    v=&\tilde{v}+be^{i\skakko{kr-\omega t}}
\end{align*}
と展開することができる。これを使うと、
\begin{align}
    f\skakko{u,v}=f\skakko{\tilde{u},\tilde{v}}+f_u\skakko{\tilde{u},\tilde{v}}ae^{i\skakko{kr-\omega t}}+f_v\skakko{\tilde{u},\tilde{v}}be^{i\skakko{kx-\omega t}}+o(a,b)
\end{align}
のようになるので、Turing方程式は
\begin{align}
    -i\omega
    \begin{pmatrix}
        a\\
        b
    \end{pmatrix}
    =
    \begin{pmatrix}
        -\abs{\mathbf{k}}^2D_u+f_u     &   f_v         \\
        g_u             &   -\abs{\mathbf{k}}^2D_v+g_v 
    \end{pmatrix}
    \begin{pmatrix}
        a\\
        b
    \end{pmatrix}
\end{align}
であり、解が存在するためには
\begin{align*}
    0
    =&
    \begin{vmatrix}
        i\omega-\abs{\mathbf{k}}^2D_u+f_u     &   f_v         \\
        g_u             &   i\omega-\abs{\mathbf{k}}^2D_v+g_v 
    \end{vmatrix}\\
    =&\skakko{i\omega-\abs{\mathbf{k}}^2D_u+f_u}\skakko{i\omega-\abs{\mathbf{k}}^2D_v+g_v}-f_vg_u\\
    =&-\omega^2+i\omega\mkakko{-\abs{\mathbf{k}}^2\skakko{D_u+D_v}+f_u+g_v}+\abs{\mathbf{k}}^4D_uD_v-\abs{\mathbf{k}}^2\skakko{D_ug_v+D_vf_u}+f_ug_v-f_vg_u\\
    =&-\omega^2+i\omega A\skakko{u,v}+B\skakko{u,v}
\end{align*}
となる$\omega$があればよく、2次方程式の解の存在条件から





\part{数値計算}\label{Numarical Analysis}
Turing反応式の一般式
\begin{align}\label{Turing_reaction_eq}
    \frac{du}{dt}=&D_u \nabla^2 u+f(u,v)\\
    \frac{dv}{dt}=&D_v \nabla^2 v+g(u,v)
\end{align}
であり、fやgを変化させることによってモデルを区別する。
今回の計算では時間を$dt$、空間を$dx$に分割することにする。すると、対応する差分方程式は
\begin{align*}
    \frac{u(t+dt,x,y)-u(t,x,y)}{dt}=&D_u\frac{u(t,x+dx,y)-2u(t,x,y)+u(t,x+dx,y)}{dx^2}\\
    +& D_u\frac{u(t,x,y+dx)-2u(t,x,y)+u(t,x,y-dx)}{dx^2}+f(u,v)\\
    \frac{v(t+dt,x,y)-v(t,x,y)}{dt}=&D_v\frac{v(t,x+dx,y)-2v(t,x,y)+v(t,x+dx,y)}{dx^2}\\
    +& D_v\frac{v(t,x,y+dx)-2v(t,x,y)+v(t,x,y-dx)}{dx^2}+f(u,v)
\end{align*}
  FitzHung-南雲モデル

今回のモデルでは
\begin{align}
    f(u,v)=&u-u^3-v\\
    g(u,v)=&γ(u-αuv-β)
\end{align}
であり、



\bibliography{ref} %hoge.bibから拡張子を外した名前
\bibliographystyle{junsrt} %参考文献出力スタイル

\end{document}